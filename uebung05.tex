\documentclass{uebblatt}

\begin{document}

\maketitle{5}

\begin{aufgabe}{Dedekindringe mit nur endlich vielen Primidealen}
Sei~$A$ ein Dedekindring, der nur endlich viele
Primideale~$\ppp_1,\ldots,\ppp_n$ besitzt. Begründe kurz:
\begin{enumerate}
\item Es gibt ein Element~$\pi \in \ppp_1 \setminus \ppp_1^2$.
\item Es gibt ein Element~$x \in A$ mit~$x \equiv \pi$
mod~$\ppp_1$ und~$x \equiv 1$ mod~$\ppp_k$ für~$k \geq 2$.
\item Für dieses Element gilt~$\ppp_1 = (x)$.
\item Alle Ideale von~$A$ sind Hauptideale.
\end{enumerate}
\end{aufgabe}

\begin{aufgabe}{Ideale und Faktorringe von Dedekindringen}
Sei~$A$ ein Dedekindring.
\begin{enumerate}
\item Sei~$\ppp \in A$ ein Primideal. Zeige, dass~$A/\ppp^n$ ein Hauptidealring
ist.
\item Sei~$\aaa \in A$ ein Ideal mit~$\aaa \neq (0)$. Zeige, dass~$A/\aaa$ ein
Hauptidealring ist.
\item Sei~$\aaa \in A$ ein Ideal. Sei~$x \in \aaa$ mit~$x \neq 0$. Zeige,
dass~$\aaa$ von zwei Elementen erzeugt werden kann, von denen eines~$x$ ist.
\end{enumerate}
{\tiny\emph{Hinweis.} In einem Faktorring~$A/\bbb$ gibt es genau so viele
Primideale, wie es Primideale in~$A$ gibt, welche~$\bbb$ umfassen.\par}
\end{aufgabe}

\begin{aufgabe}{Beispiel für eine Volumenberechnung}
Sei~$K \defeq \QQ[\sqrt{-5}]$. Sei~$\ppp \defeq (3, 1 + 2\sqrt{-5}) \subseteq
\O_K$. Bestimme das Volumen des vollständigen
Gitters~$j[\ppp] \subseteq K_\RR$, wobei~$j : K \hookrightarrow K_\RR$ die
Einbettung in den Minkowskiraum ist.
\end{aufgabe}

\begin{aufgabe}{Charakterisierung von Gittern}
Zeige, dass eine Untergruppe~$\Gamma \subseteq \RR^n$ genau dann ein Gitter
ist, wenn sie diskret ist (wenn also zu jedem Punkt~$\gamma \in \Gamma$ eine
offene Umgebung~$U \subseteq \RR^n$ von~$\gamma$ mit~$U \cap \Gamma =
\{\gamma\}$ existiert).
\end{aufgabe}

\begin{aufgabe}{Undiskretheit von Ganzheitsringen}
Sei~$K$ ein Zahlkörper vom Grad~$\geq 3$. Zeige, dass zu jedem~$\varepsilon >
0$ ein Element~$a \in \O_K \setminus \{0\}$ existiert, dessen komplexer Betrag
kleiner als~$\varepsilon$ ist.
\end{aufgabe}

\begin{aufgabe*}{Geradenbündel über dem Spektrum von Ganzheitsringen}
Sei~$A$ ein Dedekindring. Zeige: Die gebrochenen Ideale von~$K$ sind
als~$A$-Moduln projektiv.
\end{aufgabe*}

\end{document}
