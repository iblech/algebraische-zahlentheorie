\documentclass{uebblatt}

\begin{document}

\maketitle{5}

\begin{aufgabe}{Ideale und Faktorringe von Dedekindringen}
Sei~$A$ ein Dedekindring und~$\aaa \subseteq A$ ein Ideal.
\begin{enumerate}
\item Zeige: Der Faktorring $A/\aaa$ ist ein Hauptidealring, falls~$\aaa \neq
(0)$.
\item Zeige: Das Ideal~$\aaa$ lässt sich durch zwei Elemente erzeugen.
\end{enumerate}
\end{aufgabe}

\begin{aufgabe}{Beispiel für eine Volumenberechnung}
Sei~$K \defeq \QQ[\sqrt{-5}]$. Sei~$\ppp \defeq (3, 1 + 2\sqrt{-5}) \subseteq
\O_K$. Bestimme das Volumen des vollständigen
Gitters~$j[\ppp] \subseteq K_\RR$, wobei~$j : K \hookrightarrow K_\RR$ die
Einbettung in den Minkowskiraum ist.
\end{aufgabe}

\begin{aufgabe}{Charakterisierung von Gittern}
Zeige, dass eine Untergruppe~$\Gamma \subseteq \RR^n$ genau dann ein Gitter
ist, wenn sie diskret ist (wenn also zu jedem Punkt~$\gamma \in \Gamma$ eine
offene Umgebung~$U \subseteq \RR^n$ von~$\gamma$ mit~$U \cap \Gamma =
\{\gamma\}$ existiert).
\end{aufgabe}

\begin{aufgabe}{Undiskretheit von Ganzheitsringen}
Sei~$K$ ein Zahlkörper vom Grad~$\geq 3$. Zeige, dass zu jedem~$\varepsilon >
0$ ein Element~$a \in \O_K \setminus \{0\}$ existiert, dessen komplexer Betrag
kleiner als~$\varepsilon$ ist.
\end{aufgabe}

\begin{aufgabe*}{Geradenbündel über dem Spektrum von Ganzheitsringen}
Sei~$A$ ein Dedekindring. Zeige: Die gebrochenen Ideale von~$K$ sind
als~$A$-Moduln projektiv.
\end{aufgabe*}

\end{document}
