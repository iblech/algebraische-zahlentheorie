\documentclass{uebblatt}

\begin{document}

\maketitle{5}

\begin{aufgabe}{Dedekindringe mit nur endlich vielen Primidealen}
Sei~$A$ ein Dedekindring, der nur endlich viele
Primideale~$\ppp_1,\ldots,\ppp_n \neq (0)$ hat. Begründe kurz:
\begin{enumerate}
\item Es gibt ein Element~$\pi \in \ppp_1 \setminus \ppp_1^2$.
\item Es gibt ein Element~$x \in A$ mit~$x \equiv \pi$
mod~$\ppp_1$ und~$x \equiv 1$ mod~$\ppp_k$ für~$k \geq 2$.
\item Für dieses Element gilt~$\ppp_1 = (x)$.
\item Alle Ideale von~$A$ sind Hauptideale.
\end{enumerate}
\end{aufgabe}

\begin{aufgabe}{Ideale und Faktorringe von Dedekindringen}
Sei~$A$ ein Dedekindring.
\begin{enumerate}
\item Sei~$\ppp \subseteq A$ ein Primideal mit~$\ppp \neq (0)$. Sei~$m \geq 0$.
Zeige:~$A/\ppp^m$ ist ein Hauptidealring.
\item Sei~$\aaa \subseteq A$ ein Ideal mit~$\aaa \neq (0)$. Zeige, dass~$A/\aaa$ ein
Hauptidealring ist.
\item Sei~$\aaa \subseteq A$ ein Ideal. Sei~$x \in \aaa$ mit~$x \neq 0$. Zeige,
dass~$\aaa$ von zwei Elementen erzeugt werden kann, von denen eines~$x$ ist.
\end{enumerate}
{\tiny\emph{Hinweis.} In einem Faktorring~$A/\bbb$ gibt es genau so viele
Primideale, wie es Primideale in~$A$ gibt, welche~$\bbb$ umfassen. Erinnere
dich an den chinesischen Restsatz.\par}
\end{aufgabe}

\begin{aufgabe}{Beispiel für eine Volumenberechnung}
Sei~$K \defeq \QQ[\sqrt{-5}]$. Sei~$\ppp \defeq (3, 1 + 2\sqrt{-5}) \subseteq
\O_K$. Bestimme das Volumen des vollständigen
Gitters~$j[\ppp] \subseteq K_\RR$, wobei~$j : K \hookrightarrow K_\RR$ die
Einbettung in den Minkowskiraum ist.

{\tiny\emph{Bemerkung.} Es gibt eine Formel für das Volumen, die sofort den
Wert~$6 \sqrt{5}$ liefert. Aber es ist spannender, das Volumen per Hand zu
berechnen.\par}
\end{aufgabe}

\begin{aufgabe}{Charakterisierung von Gittern}
\begin{enumerate}
\item Zeige, dass eine Untergruppe~$\Gamma \subseteq \RR^n$ genau dann ein Gitter
ist, wenn sie diskret ist (wenn jeder Punkt~$\gamma \in \Gamma$ eine
offene Umgebung~$U \subseteq \RR^n$ mit~$U \cap \Gamma =
\{\gamma\}$ besitzt).

{\tiny\emph{Hinweis.} Wähle für die Rückrichtung eine maximale
über~$\RR$ linear unabhängige Familie~$(\gamma_1,\ldots,\gamma_m)$ von Vektoren
aus~$\Gamma$; setze~$\Gamma_0 \defeq
\operatorname{span}_\ZZ(\gamma_1,\ldots,\gamma_m)$; zeige, dass~$q \defeq
|\Gamma/\Gamma_0|$ endlich ist (überlege dir dazu, wie die Äquivalenzklassen
in~$\Gamma/\Gamma_0$ aussehen); folgere, dass~$q\Gamma \subseteq \Gamma_0$; und
zeige damit die Behauptung.\par}
\item Sei~$K \subseteq \CC$ ein Zahlkörper vom Grad~$\geq 3$. Folgere, dass es zu
jeder Zahl~$\varepsilon > 0$ ein Element~$a \in \O_K \setminus \{0\}$ gibt,
dessen komplexer Betrag kleiner als~$\varepsilon$ ist.
\end{enumerate}
\end{aufgabe}

\begin{aufgabe*}{Dedekindringe mit Klassenzahl 1}
Zeige, dass ein Dedekindring genau dann faktoriell ist, wenn er ein
Hauptidealbereich ist.

{\tiny\emph{Hinweis.} Zeige für die Hinrichtung, dass jedes von Null
verschiedene Primideal~$\ppp$ ein Hauptideal ist. Fixiere dazu ein Element~$\pi
\in \ppp \setminus \{0\}$ und zerlege zum einen das Ideal~$(\pi)$ in Primideale (welches
Ideal kommt dabei sicher vor?) und zum anderen das Element~$\pi$ in
Primfaktoren.\par}
\end{aufgabe*}

\begin{aufgabe*}{Geradenbündel über dem Spektrum von Ganzheitsringen}
Sei~$K$ ein Zahlkörper. Zeige: Die gebrochenen Ideale von~$K$ sind
als~$\O_K$-Moduln projektiv.
\end{aufgabe*}

\end{document}

plot "<perl -we 'for my $a (-10..10) { for my $b (-10..10) { print 3*$a+$b, chr(32), -$b*sqrt(5), $/ } }'" w p ps 10 pt 5, 0
