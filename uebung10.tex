\documentclass[entwurf]{uebblatt}
\begin{document}

\maketitle{10}

\begin{aufgabe}{Das inverse galoissche Problem im abelschen Fall}
\begin{enumerate}
\item Sei~$n$ eine positive Zahl. Finde einen Zahlkörper~$K$
mit~$\Gal(K|\QQ) \cong \ZZ/(n)$.

{\tiny\emph{Hinweis.} Finde nach Dirichlets Satz eine Primzahl~$p$ mit~$p
\equiv 1$ modulo~$n$ und konstruiere~$K$ als geeigneten Fixkörper
von~$\QQ(\zeta_p)$ über~$\QQ$.\par}

\item Sei~$A$ eine endliche abelsche Gruppe. Finde einen Zahlkörper~$K$
mit~$\Gal(K|\QQ) \cong A$.

{\tiny\emph{Hinweis.} Wir können~$A \cong \ZZ/(n_1) \times \cdots \times
\ZZ/(n_r)$ schreiben und nach Dirichlets Satz \emph{verschiedene}
Primzahlen~$p_i$ mit~$p_i \equiv 1$ modulo~$n_i$ finden. Wir können dann den
gesuchten Zahlkörper~$K$ als den Fixkörper
des Körpers~$\QQ(\zeta_{p_1}\cdots\zeta_{p_r})$ bezüglich einer geeigneten Untergruppe
seiner Galoisgruppe finden.\par}
\end{enumerate}
\end{aufgabe}

\end{document}
