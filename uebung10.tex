\documentclass[entwurf]{uebblatt}
\begin{document}

\maketitle{10}

\begin{aufgabe}{Das inverse galoissche Problem im abelschen Fall}
\begin{enumerate}
\item Sei~$n \geq 1$. Finde eine galoissche Erweiterung~$K$ von~$\QQ$
mit~$\Gal(K|\QQ) \cong \ZZ/(n)$.

{\tiny\emph{Hinweis.} Finde nach Dirichlets Satz eine Primzahl~$p$ mit~$p
\equiv 1$ modulo~$n$ und konstruiere~$K$ als geeigneten Fixkörper
von~$\QQ(\zeta_p)$ über~$\QQ$.\par}

\item Sei~$A$ eine endliche abelsche Gruppe. Finde eine galoissche Erweiterung~$K$
von~$\QQ$ mit~$\Gal(K|\QQ) \cong A$.

{\tiny\emph{Hinweis.} Wir können~$A \cong \ZZ/(n_1) \times \cdots \times
\ZZ/(n_r)$ schreiben und nach Dirichlets Satz \emph{verschiedene}
Primzahlen~$p_i$ mit~$p_i \equiv 1$ modulo~$n_i$ finden. Wir können dann die
gesuchte Erweiterung~$K$ als den Fixkörper
der Erweiterung~$\QQ(\zeta_{p_1}\cdots\zeta_{p_r})|\QQ$ bezüglich einer geeigneten Untergruppe
seiner Galoisgruppe finden. Diese ist unkanonisch isomorph zu~$\ZZ/(p_1-1)
\times \cdots \times \ZZ/(p_r-1)$.\par}

\item[{\includegraphics[height=8pt]{images/trollface} c)}]
Löse Teilaufgabe~b) für nichtkommutative endliche Gruppen.
\end{enumerate}
\end{aufgabe}

\begin{aufgabe}{Für Matthias S.}
Seien~$p$ und~$q$ Primzahlen mit~$p \neq q$. Seien~$\zeta_p$ und~$\zeta_q$
entsprechende primitive Einheitswurzeln.
\begin{enumerate}
\item[$\heartsuit$ a)] Erinnere dich, wie man für~$n \geq 1$ zeigt,
dass~$\Gal(\QQ(\zeta_n)|\QQ) \cong (\ZZ/(n))^\times$.
\item Zeige: $\QQ(\zeta_p, \zeta_q) = \QQ(\zeta_{pq})$.

{\tiny\emph{Hinweis.} Dein Beweis zeigt allgemeiner, dass~$\QQ(\zeta_n,
\zeta_m) = \QQ(\zeta_{\operatorname{kgV}(n,m)})$.\par}
\item Zeige: $\QQ(\zeta_p) \cap \QQ(\zeta_q) = \QQ$.

{\tiny\emph{Hinweis.} Auch diese Behauptung gilt allgemeiner (mit ggT statt
kgV), ist dann aber etwas komplizierter zu beweisen. Es gibt mehrere Beweise
der spezialisierten Behauptung. Interessant ist zum Beispiel folgender:
Erinnere dich, dass sich~$p$ in~$\QQ(\zeta_p)$ mit~$r = f = 1$ zerlegt. Zeige,
dass sich~$p$ in~$\QQ(\zeta_q)$ mit~$e = 1$ zerlegt. Folgere, dass sich~$p$
in~$\QQ(\zeta_p) \cap \QQ(\zeta_q)$ mit~$r = e = f = 1$ zerlegt. Wieso genügt
das?\par}
\end{enumerate}
\end{aufgabe}

\end{document}
