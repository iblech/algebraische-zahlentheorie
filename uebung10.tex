\documentclass{uebblatt}
\begin{document}

\maketitle{10}

\begin{aufgabe}{Das inverse galoissche Problem im abelschen Fall}
\begin{enumerate}
\item Sei~$n \geq 1$. Finde eine galoissche Erweiterung~$K$ von~$\QQ$
mit~$\Gal(K|\QQ) \cong \ZZ/(n)$.

{\tiny\emph{Hinweis.} Finde nach Dirichlets Satz eine Primzahl~$p$ mit~$p
\equiv 1$ modulo~$n$ und konstruiere~$K$ als geeigneten Fixkörper
von~$\QQ(\zeta_p)$ über~$\QQ$.\par}

\item Sei~$A$ eine endliche abelsche Gruppe. Finde eine galoissche Erweiterung~$K$
von~$\QQ$ mit~$\Gal(K|\QQ) \cong A$.

{\tiny\emph{Hinweis.} Wir können~$A \cong \ZZ/(n_1) \times \cdots \times
\ZZ/(n_r)$ schreiben und nach Dirichlets Satz \emph{verschiedene}
Primzahlen~$p_i$ mit~$p_i \equiv 1$ modulo~$n_i$ finden. Wir können dann die
gesuchte Erweiterung~$K$ als den Fixkörper
der Erweiterung~$\QQ(\zeta_{p_1}\cdots\zeta_{p_r})|\QQ$ bezüglich einer geeigneten Untergruppe
ihrer Galoisgruppe finden. Diese ist unkanonisch isomorph zu~$\ZZ/(p_1-1)
\times \cdots \times \ZZ/(p_r-1)$.\par}

\item[{\includegraphics[height=8pt]{images/trollface} c)}]
Löse Teilaufgabe~b) für nichtkommutative endliche Gruppen.
\end{enumerate}
\end{aufgabe}

\begin{aufgabe}{Für Matthias S.}
Seien~$p$ und~$q$ Primzahlen mit~$p \neq q$. Seien~$\zeta_p$ und~$\zeta_q$
entsprechende primitive Einheitswurzeln.
\begin{enumerate}
\item[$\heartsuit$ a)] Erinnere dich, wie man für~$n \geq 1$ zeigt,
dass~$\Gal(\QQ(\zeta_n)|\QQ) \cong (\ZZ/(n))^\times$.
\addtocounter{enumi}{1}

\item Zeige ohne viel Mühe: $\QQ(\zeta_p, \zeta_q) = \QQ(\zeta_{pq})$.

{\tiny\emph{Hinweis.} Dein Beweis zeigt allgemeiner, dass~$\QQ(\zeta_n,
\zeta_m) = \QQ(\zeta_{\operatorname{kgV}(n,m)})$.\par}
\item Zeige: $\QQ(\zeta_p) \cap \QQ(\zeta_q) = \QQ$.

{\tiny\emph{Hinweis.} Auch diese Behauptung gilt allgemeiner (mit ggT statt
kgV), ist dann aber etwas komplizierter zu beweisen. Es gibt mehrere Beweise
der spezialisierten Behauptung. Interessant ist zum Beispiel folgender:
Erinnere dich, dass sich~$p$ in~$\QQ(\zeta_p)$ mit~$r = f = 1$ zerlegt. Zeige,
dass sich~$p$ in~$\QQ(\zeta_q)$ mit~$e = 1$ zerlegt. Folgere, dass sich~$p$
in~$\QQ(\zeta_p) \cap \QQ(\zeta_q)$ mit~$r = e = f = 1$ zerlegt. Wieso genügt
das?\par}
\end{enumerate}
\end{aufgabe}

\begin{aufgabe}{Ein Kriterium für die Unmöglichkeit einer Potenzbasis}
Sei~$K$ ein Zahlkörper vom Grad~$n$. Existiere eine Primzahl~$p < n$, welche
in~$K$ voll zerlegt ist. Zeige, dass kein~$\alpha \in K$ mit~$\O_K =
\ZZ[\alpha]$ existiert.
\end{aufgabe}

\begin{aufgabe}{Endlich etwas Konzeptionelles zum Eisenstein-Kriterium}
Ein normiertes Polynom~$f(X) = X^n + a_{n-1}X^{n-1} + \cdots + a_1X + a_0 \in \ZZ[X]$
heißt genau dann \emph{Eisensteinsch} bei einer Primzahl~$p$, wenn alle~$a_i$
durch~$p$ teilbar, der konstante Koeffizient~$a_0$ aber nicht durch~$p^2$ teilbar ist.
Man lernt, dass solche Polynome stets irreduzibel sind.
\begin{enumerate}
\item Sei~$\vartheta$ eine Nullstelle eines solchen Polynoms. Zeige, dass~$p$
in~$\QQ(\vartheta)$ rein verzweigt ist.

{\tiny\emph{Tipp.} Sei~$\ppp$ einer der Primidealfaktoren von~$(p) \subseteq
\O_K$. Sei~$e$ sein Verzweigungsindex; es gilt also~$(p) \subseteq \ppp^e$ und
wir hoffen,~$e = n$ nachweisen zu können. Zeige, dass~$a_i \vartheta^i$ für~$i
= 1,\ldots,n-1$ in~$\ppp^{e+1}$ liegt. Zeige weiter, dass~$a_0$ (zwar
in~$\ppp^e$, aber) nicht in~$\ppp^{e+1}$ liegt. Folgere, dass~$\vartheta^n$
nicht in~$\ppp^{e+1}$ liegt. Beobachte, dass~$\vartheta^n$ aber in~$\ppp^n$
liegt. Sei fertig.\par}

\item Welche Primzahlen muss man also nur untersuchen, wenn man das
Eisenstein-Kriterium anwenden möchte? Ist deine Antwort
sogar robust gegen Verschiebungen des Polynoms, also dem Übergang zu~$f(X-a)$?
\end{enumerate}
\end{aufgabe}

\begin{aufgabe*}{Eine Knobelaufgabe vom Erfinders des Blogs}
Für welche Primzahlen~$p$ ist~$1/p$ ein Dezimalbruch mit Periodenlänge~$10$?
\end{aufgabe*}

\end{document}
