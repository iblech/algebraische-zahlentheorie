\documentclass[entwurf]{uebblatt}
\begin{document}

\maketitle{8}

\begin{aufgabe}{Verzweigung ist die Ausnahme}
Sei~$K$ ein Zahlkörper vom Grad~$n$. Gelte~$K = \QQ[\vartheta]$ mit~$\vartheta
\in \O_K$.
\begin{enumerate}
\item Zeige: Die Diskriminante~$d_\vartheta$
der~$\QQ$-Basis~$(1,\vartheta,\ldots,\vartheta^{n-1})$ von~$K$ ist gleich der
Diskriminante des Minimalpolynoms von~$\vartheta$.
\item Sei~$p$ eine Primzahl, sodass die Ideale~$(p)$ und~$\FFF_\vartheta$
von~$\O_K$ zueinander teilerfremd sind. Zeige, dass~$p$ genau dann in~$K$
verzweigt ist, wenn~$p \mid d_\vartheta$.
\item Zeige: Nur endlich viele Primzahlen sind in~$K$ verzweigt. Kannst du die
Kandidaten für verzweigte Primzahlen sogar explizit angeben?
\item Interpretiere Aufgabe 2 von Blatt 4 in neuem Licht.
\end{enumerate}
\end{aufgabe}

\begin{aufgabe*}{Verzweigte Überlagerungen in der komplexen Geometrie}
\begin{enumerate}
\item Informiere dich über verzweigte Überlagerungen (branched coverings) in der
komplexen Geometrie und vergleiche die dortige Situation mit der fundamentalen
Gleichung.
\item Frage Sven, was er dir zu diesem Thema auf jeden Fall mitgeben möchte.
\end{enumerate}
\end{aufgabe*}

\begin{aufgabe*}{Lücken zwischen Primzahlen}
Zeige: Zu jeder Lauflänge~$n \geq 1$ gibt es eine Folge von~$n$
aufeinanderfolgenden natürlichen Zahlen, welche alle keine Primzahlen sind.
\end{aufgabe*}

\end{document}

http://www.math.uni-hamburg.de/home/posingies/aufgaben/aztlsg5.pdf

https://web.archive.org/web/20060903142509/http://modular.math.washington.edu/129-05/notes/129.pdf
Aufgabe zu quadratischer Reziprozität
